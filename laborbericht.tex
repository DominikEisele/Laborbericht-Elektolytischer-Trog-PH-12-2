\documentclass[a4paper,12pt,fleqn,oneside]{article}
 \usepackage{graphicx}
\usepackage{etex}
\usepackage[latin1]{inputenc}
\usepackage[ngerman]{babel}
\usepackage{ae,aecompl}
\usepackage[T1]{fontenc}
\usepackage{ngerman}
%\usepackage{fleqn}
\usepackage{ulem}
\usepackage{amssymb}
\usepackage[locale=DE, per-mode=fraction, quotient-mode=fraction, group-minimum-digits=6]{siunitx}
\usepackage{tabularx}
\usepackage{bm}
\usepackage{booktabs}
\usepackage{color}
\usepackage{pictex}
\usepackage[left=2.5cm,right=2.5cm,top=2cm,bottom=2cm,includeheadfoot]{geometry}
\usepackage[section]{placeins}
\usepackage{xspace}
\usepackage{multirow}
\usepackage{lastpage}
\usepackage{fancyhdr}
\usepackage{graphicx}
\usepackage{esvect}
\usepackage{pgfplots}
\usepackage[ngerman]{babel}
\usepackage {graphics}
\usepackage {graphicx}
\usepackage{tikz}
\usepackage{amsmath}
\usepackage{autonum}


\setlength{\headheight}{15pt}
\pagestyle{fancy}
\fancyfoot[C]{Seite \thepage{} von \pageref{LastPage}}
\linespread{1.5}
\author{Dominik Eisele}
\title{Laborbericht}
\date{\today}


%\begingroup\makeatletter\ifx\SetFigFontNFSS\undefined%
%\gdef\SetFigFontNFSS#1#2#3#4#5{%
%  \reset@font\fontsize{#1}{#2pt}%
%  \fontfamily{#3}\fontseries{#4}\fontshape{#5}%
%  \selectfont}%
%\fi\endgroup%


% Zus�tzliche Spalten mit variabler Breite fur tabularx
%\newcolumntype{L}{>{\raggedright\arraybackslash}X} % linksb�ndig
%\newcolumntype{C}{>{\centering\arraybackslash}X} % zentriert
%\newcolumntype{R}{>{\raggedleft\arraybackslash}X} % rechtsb�ndig


\newcolumntype{L}[1]{>{\raggedright\arraybackslash}p{#1}} % linksb�ndig mit Breitenangabe
\newcolumntype{C}[1]{>{\centering\arraybackslash}p{#1}} % zentriert mit Breitenangabe
\newcolumntype{R}[1]{>{\raggedleft\arraybackslash}p{#1}} % rechtsb�ndig mit Breitenangabe


\setlength{\tabcolsep}{0pt}
\renewcommand{\arraystretch}{1}


%\renewcommand*\contentsname{Gliederung}


\let\oldsqrt\sqrt
\def\sqrt{\mathpalette\DHLhksqrt}
\def\DHLhksqrt#1#2{\setbox0=\hbox{$#1\oldsqrt{#2\,}$}\dimen0=\ht0
\advance\dimen0-0.3\ht0
%0.3 ist das Ma� f�r die Hakenl�nge, relativ zum Inhalt der Wurzel
\setbox2=\hbox{\vrule height\ht0 depth -\dimen0}%
{\box0\lower0.4pt\box2}}



\begin{document}

\begin{titlepage}
	\begin{flushleft}
		\vspace*{2\baselineskip}
		{\fontsize{16}{19.2}\selectfont Laborbericht Physik TGE12/2 A}\\[4\baselineskip]
		\begin{tabularx}{\textwidth}{rp{5px}X}
			{\fontsize{16}{19.2}\selectfont Titel:}&&{\fontsize{16}{19.2}\selectfont Elektrolytischer Trog}
		\end{tabularx}
		\\[5\baselineskip]
		\setlength{\tabcolsep}{0pt}
		\renewcommand{\arraystretch}{1,25}
		\begin{tabular}{lp{5px}l}
			{\fontsize{14}{16.8}\selectfont Bearbeiter:}&&{\fontsize{14}{16.8}\selectfont Dominik Eisele{}}\\
			{\fontsize{14}{16.8}\selectfont Mitarbeiterin:}&&{\fontsize{14}{16.8}\selectfont Theresa Klein}\\
			{\fontsize{14}{16.8}\selectfont Datum Versuchsdurchf�hrung:}&&{\fontsize{14}{16.8}\selectfont 16.11.2015}\\
			{\fontsize{14}{16.8}\selectfont Datum Abgabe:}&&{\fontsize{14}{16.8}\selectfont 03.12.2015}
		\end{tabular}
		\\[2\baselineskip]
		{\fontsize{14}{16.8}\selectfont Ich erkl�re an Eides statt, den vorliegenden Laborbericht selbst angefertigt zu haben. Alle fremden Quellen wurden in diesem Laborbericht benannt.}
		\\[2\baselineskip]
		{\fontsize{14}{16.8}\selectfont Aichwald, \today $  $ Dominik Eisele}
	\end{flushleft}
\end{titlepage}

\setlength{\tabcolsep}{7pt}
\renewcommand{\arraystretch}{1,7}

\newpage
\tableofcontents
\newpage


\section{Einf�hrung}
	Bei dem Versuch "`Elektrolytischer Trog"' ergibt sich der Verlauf der Feldlinien zwischen zwei Elektroden. Auf diesen Verlauf
	kommt man, indem man zwischen einem Plattenkondensator, der sich in einem Elektrolyt befindet, �quipotentiallinien(-fl�chen)
	sucht und sich diese Verl�ufe notiert. Durch diese �quipotentiallinien muss man jeweils eine orthogonale einzeichnen, und
	man erh�lt den Verlauf von Feldlinien in einem Plattenkondensator.\\
	An den Elektroden wurde eine Wechselspannung  angelegt, da sich so keine Debye-Schicht (Elektrochemische 
	Raumladungsdoppelschicht), wie bei einer Gleichspannung, ausbreiten konnte. Bei dieser Raumladungsdoppelschicht 
	w�rde zus�tzlich ca. \SI{1}{\volt} abfallen.

\subsection{Formeln}
	Betrag der elektrischen Feldst�rke:
	\[E=\frac{\text{Potentialdifferenz }\Delta\varphi}{\text{Abstand der Potentiallinien }\Delta s}\]

\newpage
\section{Material und Methoden}

\subsection{Material}
	F�r den Versuch verwendete Materialien:
	\begin{itemize}
		\item 1 $\times$ Netzteil
		\item 3 $\times$ Kabel
		\item 1 $\times$ Spannungsmesser
		\item 1 $\times$ Messspitze zur Spannungsmessung
		\item 1 $\times$ Elektrolytischer Trog mit planen Elektroden
		\item 1 $\times$ Metallring
		\item Millimeterpapier
	\end{itemize}

\subsection{Aufbau}
	Ein mit Wasser gef�llter elektrolytischer Trog wie er in Abbildung \ref{fig:skizze_elektrolytischer_trog} zu sehen ist, besteht aus
	einer Kunststoffwanne, die an beiden Seiten jeweils eine plan liegende Elektrode besitzt. Diese Elektroden sind so platziert, dass
	sie in das, als Elektrolyt verwendete, Leitungswasser hineinragen. An diese  Elektroden wurde eine Wechselspannung angelegt.
	Au�erdem wurde noch ein Multimeter installiert, sodass man mit einer Messspitze die Potentialdifferenz zwischen einer Elektrode
	und einer beliebigen Stelle im Elektrolyt messen konnte.\\
	In der zweiten Messreihe wurde noch ein Metallring in den elektrolytischen Trog gelegt, sodass das Magnetfeld durch ihn gest�rt
	wird.
	

	
	\begin{figure}[H!]
		\centering
		\scalebox{.9}{%Title: /tmp/xfig-fig074335
%%Created by: fig2dev Version 3.2 Patchlevel 5e
%%CreationDate: Wed Nov 25 21:38:03 2015
%%User: dominik@deepthought.private-site.de (Dominik Eisele)
\font\thinlinefont=cmr5
%
\begingroup\makeatletter\ifx\SetFigFont\undefined%
\gdef\SetFigFont#1#2#3#4#5{%
  \reset@font\fontsize{#1}{#2pt}%
  \fontfamily{#3}\fontseries{#4}\fontshape{#5}%
  \selectfont}%
\fi\endgroup%
\mbox{\beginpicture
\setcoordinatesystem units <1.04987cm,1.04987cm>
\unitlength=1.04987cm
\linethickness=1pt
\setplotsymbol ({\makebox(0,0)[l]{\tencirc\symbol{'160}}})
\setshadesymbol ({\thinlinefont .})
\setlinear
%
% Fig TEXT object
%
\put{Messspitze} [lB] at  5.239 20.098
%
% Fig TEXT object
%
\put{4} [lB] at -0.476 14.192
%
% Fig TEXT object
%
\put{5} [lB] at -0.476 13.240
%
% Fig TEXT object
%
\put{6} [lB] at -0.476 12.287
%
% Fig TEXT object
%
\put{7} [lB] at -0.476 11.335
%
% Fig TEXT object
%
\put{8} [lB] at -0.476 10.382
%
% Fig TEXT object
%
\put{10 V} [lB] at  4.858  7.049
%
% Fig TEXT object
%
\put{Metallring} [lB] at  6.382 12.954
%
% Fig TEXT object
%
\put{Elektroden} [lB] at  4.286 16.097
%
% Fig TEXT object
%
\put{Wassergef�llter} [lB] at 11.239 24.479
\put{elektrolytischer Trog} [lB] at 11.249 24.010
%
% Fig ELLIPSE
%
\linethickness= 0.500pt
\setplotsymbol ({\thinlinefont .})
{\color[rgb]{0,0,0}\ellipticalarc axes ratio  1.715:1.715  360 degrees 
	from  6.477 11.430 center at  4.763 11.430
}%
%
% Fig ELLIPSE
%
\linethickness= 0.500pt
\setplotsymbol ({\thinlinefont .})
{\color[rgb]{0,0,0}\put{\makebox(0,0)[l]{\circle*{ 0.191}}} at 12.383 18.288
}%
%
% Fig POLYLINE object
%
\linethickness= 0.500pt
\setplotsymbol ({\thinlinefont .})
{\color[rgb]{0,0,0}\putrule from  0.000 18.098 to 10.478 18.098
}%
%
% Fig POLYLINE object
%
\linethickness= 0.500pt
\setplotsymbol ({\thinlinefont .})
{\color[rgb]{0,0,0}\putrule from  0.953 18.288 to  0.953 17.907
}%
%
% Fig POLYLINE object
%
\linethickness= 0.500pt
\setplotsymbol ({\thinlinefont .})
{\color[rgb]{0,0,0}\putrule from  1.905 18.288 to  1.905 17.907
}%
%
% Fig POLYLINE object
%
\linethickness= 0.500pt
\setplotsymbol ({\thinlinefont .})
{\color[rgb]{0,0,0}\putrule from  2.857 18.288 to  2.857 17.907
}%
%
% Fig POLYLINE object
%
\linethickness= 0.500pt
\setplotsymbol ({\thinlinefont .})
{\color[rgb]{0,0,0}\putrule from  3.810 18.288 to  3.810 17.907
}%
%
% Fig POLYLINE object
%
\linethickness= 0.500pt
\setplotsymbol ({\thinlinefont .})
{\color[rgb]{0,0,0}\putrule from  4.763 18.288 to  4.763 17.907
}%
%
% Fig POLYLINE object
%
\linethickness= 0.500pt
\setplotsymbol ({\thinlinefont .})
{\color[rgb]{0,0,0}\putrule from  5.715 18.288 to  5.715 17.907
}%
%
% Fig POLYLINE object
%
\linethickness= 0.500pt
\setplotsymbol ({\thinlinefont .})
{\color[rgb]{0,0,0}\putrule from  6.668 18.288 to  6.668 17.907
}%
%
% Fig POLYLINE object
%
\linethickness= 0.500pt
\setplotsymbol ({\thinlinefont .})
{\color[rgb]{0,0,0}\putrule from  7.620 18.288 to  7.620 17.907
}%
%
% Fig POLYLINE object
%
\linethickness= 0.500pt
\setplotsymbol ({\thinlinefont .})
{\color[rgb]{0,0,0}\putrule from  8.572 18.288 to  8.572 17.907
}%
%
% Fig POLYLINE object
%
\linethickness= 0.500pt
\setplotsymbol ({\thinlinefont .})
{\color[rgb]{0,0,0}\putrule from  9.525 18.288 to  9.525 17.907
}%
%
% Fig POLYLINE object
%
\linethickness= 0.500pt
\setplotsymbol ({\thinlinefont .})
{\color[rgb]{0,0,0}\putrule from  0.000 18.098 to -0.191 18.098
}%
%
% Fig POLYLINE object
%
\linethickness= 0.500pt
\setplotsymbol ({\thinlinefont .})
{\color[rgb]{0,0,0}\putrule from  0.000 19.050 to -0.191 19.050
}%
%
% Fig POLYLINE object
%
\linethickness= 0.500pt
\setplotsymbol ({\thinlinefont .})
{\color[rgb]{0,0,0}\putrule from  0.000 20.003 to -0.191 20.003
}%
%
% Fig POLYLINE object
%
\linethickness= 0.500pt
\setplotsymbol ({\thinlinefont .})
{\color[rgb]{0,0,0}\putrule from  0.000 20.955 to -0.191 20.955
}%
%
% Fig POLYLINE object
%
\linethickness= 0.500pt
\setplotsymbol ({\thinlinefont .})
{\color[rgb]{0,0,0}\putrule from  0.000 21.907 to -0.191 21.907
}%
%
% Fig POLYLINE object
%
\linethickness= 0.500pt
\setplotsymbol ({\thinlinefont .})
{\color[rgb]{0,0,0}\putrule from  0.000 22.860 to -0.191 22.860
}%
%
% Fig POLYLINE object
%
\linethickness= 0.500pt
\setplotsymbol ({\thinlinefont .})
{\color[rgb]{0,0,0}\putrule from  0.000 23.812 to -0.191 23.812
}%
%
% Fig POLYLINE object
%
\linethickness= 0.500pt
\setplotsymbol ({\thinlinefont .})
{\color[rgb]{0,0,0}\putrule from  0.000 24.765 to -0.191 24.765
}%
%
% Fig POLYLINE object
%
\linethickness= 0.500pt
\setplotsymbol ({\thinlinefont .})
{\color[rgb]{0,0,0}\putrule from  0.000 25.718 to -0.191 25.718
}%
%
% Fig POLYLINE object
%
\linethickness= 0.500pt
\setplotsymbol ({\thinlinefont .})
{\color[rgb]{0,0,0}\putrule from  0.000 17.145 to -0.191 17.145
}%
%
% Fig POLYLINE object
%
\linethickness= 0.500pt
\setplotsymbol ({\thinlinefont .})
{\color[rgb]{0,0,0}\putrule from  0.000 16.192 to -0.191 16.192
}%
%
% Fig POLYLINE object
%
\linethickness= 0.500pt
\setplotsymbol ({\thinlinefont .})
{\color[rgb]{0,0,0}\putrule from  0.000 15.240 to -0.191 15.240
}%
%
% Fig POLYLINE object
%
\linethickness= 0.500pt
\setplotsymbol ({\thinlinefont .})
{\color[rgb]{0,0,0}\putrule from  0.000 14.287 to -0.191 14.287
}%
%
% Fig POLYLINE object
%
\linethickness= 0.500pt
\setplotsymbol ({\thinlinefont .})
{\color[rgb]{0,0,0}\putrule from -0.191 13.335 to  0.000 13.335
}%
%
% Fig POLYLINE object
%
\linethickness= 0.500pt
\setplotsymbol ({\thinlinefont .})
{\color[rgb]{0,0,0}\putrule from  0.000 12.383 to -0.191 12.383
}%
%
% Fig POLYLINE object
%
\linethickness= 0.500pt
\setplotsymbol ({\thinlinefont .})
{\color[rgb]{0,0,0}\putrule from  0.000 11.430 to -0.191 11.430
}%
%
% Fig POLYLINE object
%
\linethickness= 0.500pt
\setplotsymbol ({\thinlinefont .})
{\color[rgb]{0,0,0}\putrule from  0.000 10.478 to -0.191 10.478
}%
%
% Fig POLYLINE object
%
\linethickness= 0.500pt
\setplotsymbol ({\thinlinefont .})
{\color[rgb]{0,0,0}\color[rgb]{0,0,0}\putrectangle corners at  0.000 26.670 and 10.478  9.716
}%
%
% Fig POLYLINE object
%
\linethickness=2pt
\setplotsymbol ({\makebox(0,0)[l]{\tencirc\symbol{'161}}})
{\color[rgb]{0,0,0}\putrule from 10.478 22.860 to 10.478 13.716
\putrule from 10.478 13.716 to 10.478 13.621
}%
%
% Fig POLYLINE object
%
\linethickness=2pt
\setplotsymbol ({\makebox(0,0)[l]{\tencirc\symbol{'161}}})
{\color[rgb]{0,0,0}\putrule from  0.000 22.860 to  0.000 13.621
}%
%
% Fig POLYLINE object
%
\linethickness= 0.500pt
\setplotsymbol ({\thinlinefont .})
{\color[rgb]{0,0,0}\putrule from  4.763  7.811 to -1.905  7.811
}%
%
% Fig POLYLINE object
%
\linethickness= 0.500pt
\setplotsymbol ({\thinlinefont .})
{\color[rgb]{0,0,0}\putrule from  5.715  7.811 to 12.383  7.811
}%
%
% Fig POLYLINE object
%
\linethickness= 0.500pt
\setplotsymbol ({\thinlinefont .})
{\color[rgb]{0,0,0}\putrule from -1.905 18.288 to -1.905  7.811
}%
%
% Fig POLYLINE object
%
\linethickness= 0.500pt
\setplotsymbol ({\thinlinefont .})
{\color[rgb]{0,0,0}\putrule from  0.000 18.288 to -1.905 18.288
}%
%
% Fig POLYLINE object
%
\linethickness= 0.500pt
\setplotsymbol ({\thinlinefont .})
{\color[rgb]{0,0,0}\plot  6.287 12.954  6.001 12.764 /
}%
%
% Fig POLYLINE object
%
\linethickness= 0.500pt
\setplotsymbol ({\thinlinefont .})
{\color[rgb]{0,0,0}\putrule from  0.191 16.192 to  4.191 16.192
}%
%
% Fig POLYLINE object
%
\linethickness= 0.500pt
\setplotsymbol ({\thinlinefont .})
{\color[rgb]{0,0,0}\putrule from  6.191 16.192 to 10.192 16.192
}%
%
% Fig POLYLINE object
%
\linethickness= 0.500pt
\setplotsymbol ({\thinlinefont .})
{\color[rgb]{0,0,0}\plot 10.573 23.241 11.239 24.384 /
}%
%
% Fig POLYLINE object
%
\linethickness= 0.500pt
\setplotsymbol ({\thinlinefont .})
{\color[rgb]{0,0,0}\putrule from 10.478 18.288 to 12.383 18.288
\putrule from 12.383 18.288 to 12.383  7.811
}%
%
% Fig POLYLINE object
%
\linethickness= 0.500pt
\setplotsymbol ({\thinlinefont .})
{\color[rgb]{0,0,0}\putrule from 12.383 18.288 to 12.383 19.622
}%
%
% Fig POLYLINE object
%
\linethickness= 0.500pt
\setplotsymbol ({\thinlinefont .})
{\color[rgb]{0,0,0}\putrule from 12.383 20.574 to 12.383 20.669
\putrule from 12.383 20.669 to 12.383 20.765
\putrule from 12.383 20.765 to 12.383 20.860
\putrule from 12.383 20.860 to 12.383 20.955
\putrule from 12.383 20.955 to 12.383 21.050
\putrule from 12.383 21.050 to 12.383 21.145
\putrule from 12.383 21.145 to 12.383 21.241
\putrule from 12.383 21.241 to 12.383 21.336
\putrule from 12.383 21.336 to 12.383 21.431
\putrule from 12.383 21.431 to 12.383 21.526
\putrule from 12.383 21.526 to 12.383 21.622
\putrule from 12.383 21.622 to 12.383 21.717
\putrule from 12.383 21.717 to 12.383 21.812
\putrule from 12.383 21.812 to 12.383 21.907
}%
%
% Fig POLYLINE object
%
\linethickness= 0.500pt
\setplotsymbol ({\thinlinefont .})
{\color[rgb]{0,0,0}\plot 12.383 21.907  7.811 19.622 /
%
% arrow head
%
\plot  8.009 19.792  7.811 19.622  8.066 19.678 /
%
}%
%
% Fig POLYLINE object
%
\linethickness= 0.500pt
\setplotsymbol ({\thinlinefont .})
{\color[rgb]{0,0,0}\plot  7.048 20.098  7.811 19.717 /
}%
%
% Fig TEXT object
%
\put{1} [lB] at  0.857 17.431
%
% Fig TEXT object
%
\put{2} [lB] at  1.810 17.431
%
% Fig TEXT object
%
\put{3} [lB] at  2.762 17.431
%
% Fig TEXT object
%
\put{4} [lB] at  3.715 17.431
%
% Fig TEXT object
%
\put{5} [lB] at  4.667 17.431
%
% Fig TEXT object
%
\put{6} [lB] at  5.620 17.431
%
% Fig TEXT object
%
\put{7} [lB] at  6.572 17.431
%
% Fig TEXT object
%
\put{8} [lB] at  7.525 17.431
%
% Fig TEXT object
%
\put{9} [lB] at  8.477 17.431
%
% Fig TEXT object
%
\put{10} [lB] at  9.335 17.431
%
% Fig TEXT object
%
\put{1} [lB] at -0.476 18.955
%
% Fig TEXT object
%
\put{2} [lB] at -0.476 19.907
%
% Fig TEXT object
%
\put{4} [lB] at -0.476 21.812
%
% Fig TEXT object
%
\put{5} [lB] at -0.476 22.765
%
% Fig TEXT object
%
\put{3} [lB] at -0.476 20.860
%
% Fig TEXT object
%
\put{0} [lB] at -0.476 18.002
%
% Fig TEXT object
%
\put{6} [lB] at -0.476 23.717
%
% Fig TEXT object
%
\put{7} [lB] at -0.476 24.670
%
% Fig TEXT object
%
\put{8} [lB] at -0.476 25.622
%
% Fig TEXT object
%
\put{1} [lB] at -0.476 17.050
%
% Fig TEXT object
%
\put{2} [lB] at -0.476 16.097
%
% Fig TEXT object
%
\put{3} [lB] at -0.476 15.145
%
% Fig CIRCULAR ARC object
%
\linethickness= 0.500pt
\setplotsymbol ({\thinlinefont .})
{\color[rgb]{0,0,0}\circulararc 180.000 degrees from  5.207  7.811 center at  5.080  7.811
}%
%
% Fig ELLIPSE
%
\linethickness= 0.500pt
\setplotsymbol ({\thinlinefont .})
{\color[rgb]{0,0,0}\ellipticalarc axes ratio  0.381:0.381  360 degrees 
	from  5.588  7.811 center at  5.207  7.811
}%
%
% Fig POLYLINE object
%
\linethickness= 0.500pt
\setplotsymbol ({\thinlinefont .})
{\color[rgb]{0,0,0}\putrule from  5.842  7.811 to  5.588  7.811
}%
%
% Fig POLYLINE object
%
\linethickness= 0.500pt
\setplotsymbol ({\thinlinefont .})
{\color[rgb]{0,0,0}\putrule from  4.572  7.811 to  4.826  7.811
}%
%
% Fig ELLIPSE
%
\linethickness= 0.500pt
\setplotsymbol ({\thinlinefont .})
{\color[rgb]{0,0,0}\ellipticalarc axes ratio  0.286:0.286  360 degrees 
	from 12.668 20.098 center at 12.383 20.098
}%
%
% Fig POLYLINE object
%
\linethickness= 0.500pt
\setplotsymbol ({\thinlinefont .})
{\color[rgb]{0,0,0}\putrule from 12.383 19.812 to 12.383 19.622
}%
%
% Fig POLYLINE object
%
\linethickness= 0.500pt
\setplotsymbol ({\thinlinefont .})
{\color[rgb]{0,0,0}\plot 12.287 20.214 12.383 20.003 /
\plot 12.383 20.003 12.478 20.214 /
}%
%
% Fig POLYLINE object
%
\linethickness= 0.500pt
\setplotsymbol ({\thinlinefont .})
{\color[rgb]{0,0,0}\putrule from 12.383 20.384 to 12.383 20.574
}%
%
% Fig CIRCULAR ARC object
%
\linethickness= 0.500pt
\setplotsymbol ({\thinlinefont .})
{\color[rgb]{0,0,0}\circulararc 180.000 degrees from  5.207  7.811 center at  5.334  7.811
}%
\linethickness=0pt
\putrectangle corners at -1.930 26.695 and 12.685  7.017
\endpicture}
}
		\caption{Skizze des Versuchsaufbaus}
		\label{fig:skizze_elektrolytischer_trog}
	\end{figure}

\newpage
\subsection{Durchf�hrung}
	Nachdem der Versuch aufgebaut wurde, wurden mit der Messspitze �quipotentiallinien im elektrischen Feld gesucht. Diese 
	�quipotentiallinien wurden auf Millimeterpapier �bertragen, die Spannung betrug hier dabei \SI{2}{\volt}, \SI{3}{\volt},
	\SI{4}{\volt}, \SI{5}{\volt}, \SI{6}{\volt}, \SI{7}{\volt} und \SI{8}{\volt}.\\
	Anschlie�end wurden die Magnetfeldlinien eingezeichnet und es wurde die Feldst�rke $E$ f�r drei Punkte berechnet.
	Der Versuch wurde daraufhin mit einem, sich im elektrolytischem Trog befindenden, Metallring wiederholt.


\newpage
\section{Messwerte}
	In Abbildung \ref{fig:Diagramm1} sind die Messpunkte und die �quipotentiallinien f�r \SI{2}{\volt} (schwarz), \SI{3}{\volt} (pink),
	\SI{4}{\volt} (blau), \SI{5}{\volt} (hellgr�n), \SI{6}{\volt} (rot), \SI{7}{\volt} (lila) und \SI{8}{\volt} (orange) eingezeichnet.
	Dazu wurden au�erdem die magnetischen Feldlinien (dunkelgr�n) eingezeichnet, die zwischen den beiden Kondensatorplatten,
	beim Anlegen einer Spannung, entstehen.\\
	In Abbildung \ref{fig:Diagramm2} sind ebenfalls die Messpunkte, �quipotentiallinien und Feldlinien eingezeichnet, die Farben entsprechen
	den aus Abbildung \ref{fig:Diagramm1}. Hinzu kommt in Abbildung \ref{fig:Diagramm2} der Metallring, der im Inneren des Rings
	eine �quipotentialfl�che besa�, deren Spannung \SI{4.3}{\volt} betrug. Der Metallring selbst ist ebenfalls eine �quipotentiallinie,
	dessen Spannung \SI{4.7}{\volt} betrug.
			

	
	\begin{figure}
		\begin{tikzpicture}[xscale=1.2,yscale=1.2]
		
%test���
		\definecolor{cyan}{HTML}{0000FF}
		\definecolor{pink}{HTML}{F52887}
		\definecolor{green}{HTML}{00FF00}
		\definecolor{purple}{HTML}{461B7E}
		\definecolor{taupe}{HTML}{347235}
		\definecolor{grey}{HTML}{B6B6B4}

		
		% Gitter
		\foreach \x in {0.1,0.2,...,11} \draw[- ,grey ,thin] (\x,-8.8) -- (\x,9);
		\foreach \y in {-8.8,-8.7,...,9} \draw[- ,grey ,thin] (0,\y) -- (11,\y);
		\foreach \x in {0.5,1.0,...,11} \draw[- ,gray ,thin] (\x,-8.8) -- (\x,9);
		\foreach \x in {1.0,...,11} \draw[- ,black ,thin] (\x,-8.8) -- (\x,9);
		\foreach \y in {-8.5,-8,...,9} \draw[- ,gray ,thin] (0,\y) -- (11,\y);
		\foreach \y in {-8,...,9} \draw[- ,black ,thin] (0,\y) -- (11,\y);
		% Achsen
		\draw[- ,thick] (0,0) -- (11,0) node[below right] {};
		\draw[- ,thick] (0,-8.8) -- (0,9) node[left] {};
		\draw[- ,thick] (0,9) -- (11,9) node[left] {};
		\draw[- ,thick] (11,-8.8) -- (11,9) node[left] {};
		\draw[- ,thick] (0,-8.8) -- (11,-8.8) node[left] {};
		% Achsbeschriftung
		\foreach \x in {,1,...,10} \draw (\x,0.05) -- (\x,-0.05) node [below] {\x};
		\foreach \y in {-8,...,8} \draw (0.1,\y) -- (-0.1,\y) node [left] {\y};
		% Funktion
		%\draw[red, samples=50, domain=1.000244714:7] plot (\x, {sqrt((2*1.37)/(sin(\x )*9.81))});
		% Funktionbeschriftung
		%\node (11) at (7 , 1.513887287) [below] {$t=\sqrt{\frac{2s}{\sin{(\alpha)}\cdot g}}$};	
		% Messpunkte
		
		% 2V		
			\node[black]  (11) at (1.00,6.00) {\textsf{+}};
			\node[black]  (21) at (1.40,5.50) {\textsf{+}};
			\node[black]  (31) at (1.80,5.00) {\textsf{+}};
			\node[black]  (41) at (1.90,4.50) {\textsf{+}};
			\node[black]  (51) at (2.00,4.00) {\textsf{+}};
			\node[black]  (61) at (2.10,3.50) {\textsf{+}};
			\node[black]  (71) at (2.15,3.00) {\textsf{+}};
			\node[black]  (81) at (2.20,2.50) {\textsf{+}};
			\node[black]  (91) at (2.20,2.00) {\textsf{+}};
			\node[black] (101) at (2.20,1.50) {\textsf{+}};
			\node[black] (111) at (2.20,1.00) {\textsf{+}};
			\node[black] (121) at (2.30,0.50) {\textsf{+}};
			\node[black] (131) at (2.30,0.00) {\textsf{+}};
			
			\node[black] (141) at (0.80,-5.50) {\textsf{+}};
			\node[black] (151) at (1.30,-5.00) {\textsf{+}};
			\node[black] (161) at (1.70,-4.50) {\textsf{+}};
			\node[black] (171) at (1.90,-4.00) {\textsf{+}};
			\node[black] (181) at (2.05,-3.50) {\textsf{+}};
			\node[black] (191) at (2.10,-3.00) {\textsf{+}};
			\node[black] (201) at (2.20,-2.50) {\textsf{+}};
			\node[black] (211) at (2.20,-2.00) {\textsf{+}};
			\node[black] (221) at (2.20,-1.50) {\textsf{+}};
			\node[black] (231) at (2.20,-1.00) {\textsf{+}};
			\node[black] (241) at (2.30,-0.50) {\textsf{+}};

			\draw [black] plot [smooth] coordinates {(0,6.8) (1.7,5) (2.15,3) (2.3,0) (2.1,-3) (1.3,-5) (0,-6)};
		
		% 3V		
			\node[pink]  (12) at (0.70,9.00) {\textsf{+}};
			\node[pink]  (22) at (1.20,8.50) {\textsf{+}};
			\node[pink]  (32) at (1.60,8.00) {\textsf{+}};
			\node[pink]  (42) at (2.00,7.50) {\textsf{+}};
			\node[pink]  (52) at (2.40,7.00) {\textsf{+}};
			\node[pink]  (62) at (2.70,6.50) {\textsf{+}};
			\node[pink]  (12) at (3.00,6.00) {\textsf{+}};
			\node[pink]  (22) at (3.10,5.50) {\textsf{+}};
			\node[pink]  (32) at (3.30,5.00) {\textsf{+}};
			\node[pink]  (42) at (3.40,4.50) {\textsf{+}};
			\node[pink]  (52) at (3.40,4.00) {\textsf{+}};
			\node[pink]  (62) at (3.50,3.50) {\textsf{+}};
			\node[pink]  (72) at (3.50,3.00) {\textsf{+}};
			\node[pink]  (82) at (3.50,2.50) {\textsf{+}};
			\node[pink]  (92) at (3.50,2.00) {\textsf{+}};
			\node[pink] (102) at (3.60,1.50) {\textsf{+}};
			\node[pink] (112) at (3.60,1.00) {\textsf{+}};
			\node[pink] (122) at (3.60,0.50) {\textsf{+}};
			\node[pink] (132) at (3.40,0.00) {\textsf{+}};
			
			
			\node[pink] (142) at (1.50,-8.00) {\textsf{+}};
			\node[pink] (152) at (1.80,-7.50) {\textsf{+}};
			\node[pink] (162) at (1.90,-7.00) {\textsf{+}};
			\node[pink] (172) at (2.70,-6.50) {\textsf{+}};
			\node[pink] (182) at (2.60,-6.00) {\textsf{+}};
			\node[pink] (192) at (2.90,-5.50) {\textsf{+}};
			\node[pink] (202) at (3.00,-5.00) {\textsf{+}};
			\node[pink] (212) at (3.30,-4.50) {\textsf{+}};
			\node[pink] (222) at (3.40,-4.00) {\textsf{+}};
			\node[pink] (232) at (3.40,-3.50) {\textsf{+}};
			\node[pink] (242) at (3.50,-3.00) {\textsf{+}};
			\node[pink] (252) at (3.60,-2.50) {\textsf{+}};
			\node[pink] (262) at (3.50,-2.00) {\textsf{+}};
			\node[pink] (272) at (3.60,-1.50) {\textsf{+}};
			\node[pink] (282) at (3.60,-1.00) {\textsf{+}};
			\node[pink] (292) at (3.50,-0.50) {\textsf{+}};

			\draw [pink] plot [smooth] coordinates {(0.7,9) (2,7.5) (3,6) (3.25,5.5) (3.4,4.5) (3.6,0) (3.4,-4) (2.1,-7.5) (0.3,-8.8)};

		% 4V		
			\node[cyan]  (13) at (4.30,9.00) {\textsf{+}};
			\node[cyan]  (23) at (4.50,8.50) {\textsf{+}};
			\node[cyan]  (33) at (4.50,8.00) {\textsf{+}};
			\node[cyan]  (43) at (4.60,7.50) {\textsf{+}};
			\node[cyan]  (53) at (4.60,7.00) {\textsf{+}};
			\node[cyan]  (63) at (4.80,6.50) {\textsf{+}};
			\node[cyan]  (73) at (4.90,6.00) {\textsf{+}};
			\node[cyan]  (83) at (4.90,5.50) {\textsf{+}};
			\node[cyan]  (93) at (4.90,5.00) {\textsf{+}};
			\node[cyan] (103) at (4.90,4.50) {\textsf{+}};
			\node[cyan] (113) at (4.90,4.00) {\textsf{+}};
			\node[cyan] (123) at (4.90,3.50) {\textsf{+}};
			\node[cyan] (133) at (4.90,3.00) {\textsf{+}};
			\node[cyan] (143) at (4.90,2.50) {\textsf{+}};
			\node[cyan] (153) at (4.90,2.00) {\textsf{+}};
			\node[cyan] (163) at (4.90,1.50) {\textsf{+}};
			\node[cyan] (173) at (4.90,1.00) {\textsf{+}};
			\node[cyan] (183) at (4.80,0.50) {\textsf{+}};
			\node[cyan] (193) at (4.80,0.00) {\textsf{+}};
			
			\node[cyan] (203) at (4.20,-8.80) {\textsf{+}};
			\node[cyan] (213) at (4.30,-8.50) {\textsf{+}};
			\node[cyan] (223) at (4.30,-8.00) {\textsf{+}};
			\node[cyan] (233) at (4.50,-7.50) {\textsf{+}};
			\node[cyan] (243) at (4.55,-7.00) {\textsf{+}};
			\node[cyan] (253) at (4.60,-6.50) {\textsf{+}};
			\node[cyan] (263) at (4.65,-6.00) {\textsf{+}};
			\node[cyan] (273) at (4.70,-5.50) {\textsf{+}};
			\node[cyan] (283) at (4.80,-5.00) {\textsf{+}};
			\node[cyan] (293) at (4.80,-4.50) {\textsf{+}};
			\node[cyan] (303) at (4.80,-4.00) {\textsf{+}};
			\node[cyan] (313) at (4.85,-3.50) {\textsf{+}};
			\node[cyan] (323) at (4.90,-3.00) {\textsf{+}};
			\node[cyan] (333) at (4.90,-2.50) {\textsf{+}};
			\node[cyan] (343) at (5.00,-2.00) {\textsf{+}};
			\node[cyan] (353) at (5.00,-1.50) {\textsf{+}};
			\node[cyan] (363) at (5.00,-1.00) {\textsf{+}};
			\node[cyan] (373) at (5.00,-0.50) {\textsf{+}};

			\draw [cyan] plot [smooth] coordinates {(4.3,9) (4.7,8) (4.8,7.5) (5,6) (5,0) (4.8,-5) (4.3,-8) (4.1,-8.8)};

		% 5V	
			\node[green]  (14) at (6.90,9.00) {\textsf{+}};
			\node[green]  (24) at (6.80,8.50) {\textsf{+}};
			\node[green]  (34) at (6.60,8.00) {\textsf{+}};
			\node[green]  (44) at (6.50,7.50) {\textsf{+}};
			\node[green]  (54) at (6.50,7.00) {\textsf{+}};
			\node[green]  (64) at (6.40,6.50) {\textsf{+}};	
			\node[green]  (74) at (6.40,6.00) {\textsf{+}};
			\node[green]  (84) at (6.40,5.50) {\textsf{+}};
			\node[green]  (94) at (6.40,5.00) {\textsf{+}};
			\node[green] (104) at (6.30,4.50) {\textsf{+}};
			\node[green] (114) at (6.20,4.00) {\textsf{+}};
			\node[green] (124) at (6.20,3.50) {\textsf{+}};
			\node[green] (134) at (6.20,3.00) {\textsf{+}};
			\node[green] (144) at (6.10,2.50) {\textsf{+}};
			\node[green] (154) at (6.10,2.00) {\textsf{+}};
			\node[green] (164) at (6.10,1.50) {\textsf{+}};
			\node[green] (174) at (6.10,1.00) {\textsf{+}};
			\node[green] (184) at (6.10,0.50) {\textsf{+}};
			\node[green] (194) at (6.10,0.00) {\textsf{+}};
			
			\node[green] (204) at (6.20,-0.50) {\textsf{+}};
			\node[green] (214) at (6.30,-1.00) {\textsf{+}};
			\node[green] (224) at (6.30,-1.50) {\textsf{+}};
			\node[green] (234) at (6.30,-2.00) {\textsf{+}};
			\node[green] (244) at (6.30,-2.50) {\textsf{+}};
			\node[green] (254) at (6.30,-3.00) {\textsf{+}};
			\node[green] (264) at (6.40,-3.50) {\textsf{+}};
			\node[green] (274) at (6.30,-4.00) {\textsf{+}};
			\node[green] (284) at (6.50,-4.50) {\textsf{+}};
			\node[green] (294) at (6.50,-5.00) {\textsf{+}};
			\node[green] (304) at (6.50,-5.50) {\textsf{+}};
			\node[green] (314) at (6.60,-6.00) {\textsf{+}};
			\node[green] (324) at (6.60,-6.50) {\textsf{+}};
			\node[green] (334) at (6.60,-7.00) {\textsf{+}};
			\node[green] (344) at (6.70,-7.50) {\textsf{+}};
			\node[green] (354) at (6.90,-8.00) {\textsf{+}};
			\node[green] (364) at (7.00,-8.50) {\textsf{+}};
			\node[green] (374) at (7.10,-8.80) {\textsf{+}};
		
			\draw [green] plot [smooth] coordinates {(6.9,9) (6.5,8) (6.2,0) (6.6,-7) (6.75,-8) (7.1,-8.8)};

		% 6V		
			\node (05) at (11,8.55) {};
			\node[red]  (15) at (10.6,8.50) {\textsf{+}};
			\node[red]  (25) at (9.60,8.00) {\textsf{+}};
			\node[red]  (35) at (9.40,7.50) {\textsf{+}};
			\node[red]  (45) at (8.80,7.00) {\textsf{+}};
			\node[red]  (55) at (8.55,6.50) {\textsf{+}};			
			\node[red]  (65) at (8.10,6.00) {\textsf{+}};
			\node[red]  (75) at (8.00,5.50) {\textsf{+}};
			\node[red]  (85) at (7.80,5.00) {\textsf{+}};
			\node[red]  (95) at (7.70,4.50) {\textsf{+}};
			\node[red] (105) at (7.70,4.00) {\textsf{+}};
			\node[red] (115) at (7.60,3.50) {\textsf{+}};
			\node[red] (125) at (7.50,3.00) {\textsf{+}};
			\node[red] (135) at (7.40,2.50) {\textsf{+}};
			\node[red] (145) at (7.40,2.00) {\textsf{+}};
			\node[red] (155) at (7.40,1.50) {\textsf{+}};
			\node[red] (165) at (7.40,1.00) {\textsf{+}};
			\node[red] (175) at (7.30,0.50) {\textsf{+}};
			\node[red] (185) at (7.40,0.00) {\textsf{+}};
			
			\node[red] (195) at (11.0,-7.50) {\textsf{+}};
			\node[red] (205) at (8.60,-7.00) {\textsf{+}};
			\node[red] (215) at (8.40,-6.50) {\textsf{+}};
			\node[red] (225) at (8.60,-6.00) {\textsf{+}};
			\node[red] (235) at (8.50,-5.50) {\textsf{+}};
			\node[red] (245) at (8.00,-5.00) {\textsf{+}};
			\node[red] (255) at (7.90,-4.50) {\textsf{+}};
			\node[red] (265) at (7.70,-4.00) {\textsf{+}};
			\node[red] (275) at (7.60,-3.50) {\textsf{+}};
			\node[red] (285) at (7.60,-3.00) {\textsf{+}};
			\node[red] (295) at (7.60,-2.50) {\textsf{+}};
			\node[red] (305) at (7.50,-2.00) {\textsf{+}};
			\node[red] (315) at (7.50,-1.50) {\textsf{+}};
			\node[red] (325) at (7.50,-1.00) {\textsf{+}};
			\node[red] (335) at (7.40,-0.50) {\textsf{+}};

			\draw [red] plot [smooth] coordinates {(11,8.55) (8.8,7) (7.6,4) (7.5,0) (7.6,-4) (9,-6.5) (11,-7.5)};
			
		% 7V		
			\node[purple] (25) at (8.60,0.00) {\textsf{+}};
			\node[purple] (26) at (10.0,-5.5) {\textsf{+}};

			\draw [purple] plot [smooth] coordinates {(11,6.3) (9.3,5) (8.7,3) (8.5,0) (8.6,-3) (9,-4.5) (10,-5.5) (11,-6)};

		% 8V		
			\node[orange] (25) at (10.0,0.00) {\textsf{+}};
			\node[orange] (26) at (10.5,-3.5) {\textsf{+}};

			\draw [orange] plot [smooth] coordinates {(11,4) (10.2,3) (10,0) (10.15,-3) (10.5,-3.5) (11,-3.8)};

		% Feldlinien
			\draw[- ,thick ,taupe] (0,-0.75) -- (11,-0.75);
			\draw[- ,thick ,taupe] (0,1.05) -- (11,1.05);
			\draw[- ,thick ,taupe] (0,1.35) -- (11,1.35);
			\draw[- ,thick ,taupe] (0,-2.85) -- (11,-2.85);
			\draw[- ,thick ,taupe] (0,-2.05) -- (11,-2.05);
			\draw[- ,thick ,taupe] (0,2.05) -- (11,2.05);

	\end{tikzpicture}
	\caption{Messwerte}
	\label{fig:Diagramm1}

	\end{figure}

	\begin{figure}
		\begin{tikzpicture}[xscale=1.2,yscale=1.2]
		
		\definecolor{cyan}{HTML}{0000FF}
		\definecolor{pink}{HTML}{F52887}
		\definecolor{green}{HTML}{00FF00}
		\definecolor{purple}{HTML}{461B7E}
		\definecolor{taupe}{HTML}{347235}
		\definecolor{grey}{HTML}{B6B6B4}

		
		% Gitter
		\foreach \x in {0.1,0.2,...,11} \draw[- ,grey ,thin] (\x,-8.8) -- (\x,9);
<<<<<<< HEAD
		\foreach \y in {-8.8,...,9} \draw[- ,grey ,thin] (0,\y) -- (11,\y);
=======
		\foreach \y in {-8.8,-8.7,...,9} \draw[- ,grey ,thin] (0,\y) -- (11,\y);
>>>>>>> parent of bc69bd2... Revert "messreihen für 2V,3V,4V,5V"
		\foreach \x in {0.5,1.0,...,11} \draw[- ,gray ,thin] (\x,-8.8) -- (\x,9);
		\foreach \x in {1.0,...,11} \draw[- ,black ,thin] (\x,-8.8) -- (\x,9);
		\foreach \y in {-8.5,-8,...,9} \draw[- ,gray ,thin] (0,\y) -- (11,\y);
		\foreach \y in {-8,...,9} \draw[- ,black ,thin] (0,\y) -- (11,\y);
		% Achsen
		\draw[- ,thick] (0,0) -- (11,0) node[below right] {};
		\draw[- ,thick] (0,-8.8) -- (0,9) node[left] {};
		\draw[- ,thick] (0,9) -- (11,9) node[left] {};
		\draw[- ,thick] (11,-8.8) -- (11,9) node[left] {};
		\draw[- ,thick] (0,-8.8) -- (11,-8.8) node[left] {};
		% Achsbeschriftung
		\foreach \x in {,1,...,10} \draw (\x,0.05) -- (\x,-0.05) node [below] {\x};
		\foreach \y in {-8,...,8} \draw (0.1,\y) -- (-0.1,\y) node [left] {\y};
		% Funktion
		%\draw[red, samples=50, domain=1.000244714:7] plot (\x, {sqrt((2*1.37)/(sin(\x )*9.81))});
		% Funktionbeschriftung
		%\node (11) at (7 , 1.513887287) [below] {$t=\sqrt{\frac{2s}{\sin{(\alpha)}\cdot g}}$};	
		% Messpunkte
		
		% Metallring
			\draw (5,-7) circle (1.8);

		% 2V		
			\node[black]  (11) at (0.80,6.00) {\textsf{+}};		
			\node[black]  (31) at (1.50,5.00) {\textsf{+}};		
			\node[black]  (51) at (2.00,4.00) {\textsf{+}};	
			\node[black]  (71) at (2.10,3.00) {\textsf{+}};		
			\node[black]  (91) at (2.10,2.00) {\textsf{+}};
			\node[black] (111) at (2.20,1.00) {\textsf{+}};
			\node[black] (131) at (2.30,0.00) {\textsf{+}};
			
			\node[black] (141) at (0.80,-5.00) {\textsf{+}};
			\node[black] (151) at (1.60,-4.00) {\textsf{+}};
			\node[black] (171) at (1.90,-3.00) {\textsf{+}};
			\node[black] (191) at (2.20,-2.00) {\textsf{+}};
			\node[black] (211) at (2.30,-1.00) {\textsf{+}};
			
			\draw [black] plot [smooth] coordinates {(0,6.6) (1.5,5) (2.1,3) (2.3,0) (1.9,-3) (0.8,-5) (0,-5.5)};
		
		% 3V		
			\node[pink]  (12) at (0.50,9.00) {\textsf{+}};
			\node[pink]  (32) at (1.50,8.00) {\textsf{+}};
			\node[pink]  (52) at (2.10,7.00) {\textsf{+}};
			\node[pink]  (12) at (2.70,6.00) {\textsf{+}};
			\node[pink]  (32) at (3.20,5.00) {\textsf{+}};
			\node[pink]  (52) at (3.30,4.00) {\textsf{+}};
			\node[pink]  (72) at (3.50,3.00) {\textsf{+}};
			\node[pink]  (92) at (3.50,2.00) {\textsf{+}};
			\node[pink] (112) at (3.50,1.00) {\textsf{+}};
			\node[pink] (132) at (3.50,0.00) {\textsf{+}};
			
			
			\node[pink] (182) at (2.00,-6.00) {\textsf{+}};
			\node[pink] (202) at (2.30,-5.00) {\textsf{+}};
			\node[pink] (222) at (2.80,-4.00) {\textsf{+}};
			\node[pink] (242) at (3.30,-3.00) {\textsf{+}};
			\node[pink] (262) at (3.50,-2.00) {\textsf{+}};
			\node[pink] (282) at (3.50,-1.00) {\textsf{+}};

			\draw [pink] plot [smooth] coordinates {(0.5,9) (2.7,6) (3.4,4) (3.5,0) (3.5,-2)(3,-4) (1.8,-6) (0,-7.5)};

		% 4V		
			\node[cyan]  (13) at (4.10,9.00) {\textsf{+}};
			\node[cyan]  (33) at (4.40,8.00) {\textsf{+}};
			\node[cyan]  (53) at (4.50,7.00) {\textsf{+}};
			\node[cyan]  (73) at (4.60,6.00) {\textsf{+}};
			\node[cyan]  (93) at (4.70,5.00) {\textsf{+}};
			\node[cyan] (113) at (4.70,4.00) {\textsf{+}};
			\node[cyan] (133) at (4.70,3.00) {\textsf{+}};
			\node[cyan] (153) at (4.70,2.00) {\textsf{+}};
			\node[cyan] (173) at (4.70,1.00) {\textsf{+}};
			\node[cyan] (193) at (4.80,0.00) {\textsf{+}};

			\node[cyan] (303) at (4.60,-4.00) {\textsf{+}};
			\node[cyan] (323) at (4.90,-3.00) {\textsf{+}};
			\node[cyan] (343) at (4.90,-2.00) {\textsf{+}};
			\node[cyan] (363) at (4.90,-1.00) {\textsf{+}};

			\draw [cyan] plot [smooth] coordinates {(4.1,9) (4.4,8) (4.7,6) (4.8,0) (4.9,-3) (4.6,-4) (3.9,-5) (3.2,-6) (2.9,-7) (2.7,-8.8)};

		% 5V	
			\node[green]  (14) at (7.80,9.00) {\textsf{+}};
			\node[green]  (34) at (7.10,8.00) {\textsf{+}};
			\node[green]  (54) at (6.50,7.00) {\textsf{+}};
			\node[green]  (74) at (6.30,6.00) {\textsf{+}};
			\node[green]  (94) at (6.20,5.00) {\textsf{+}};
			\node[green] (114) at (6.20,4.00) {\textsf{+}};
			\node[green] (134) at (6.20,3.00) {\textsf{+}};
			\node[green] (154) at (6.10,2.00) {\textsf{+}};
			\node[green] (174) at (6.10,1.00) {\textsf{+}};
			\node[green] (194) at (6.20,0.00) {\textsf{+}};
			
			\node[green] (214) at (6.20,-1.00) {\textsf{+}};
			\node[green] (234) at (6.20,-2.00) {\textsf{+}};
			\node[green] (254) at (6.35,-3.00) {\textsf{+}};
			\node[green] (274) at (6.50,-4.00) {\textsf{+}};
			\node[green] (214) at (6.70,-5.00) {\textsf{+}};
			\node[green] (234) at (7.20,-6.00) {\textsf{+}};
			\node[green] (254) at (7.60,-7.00) {\textsf{+}};
			\node[green] (274) at (7.80,-8.00) {\textsf{+}};
			\node[green] (274) at (8.00,-8.80) {\textsf{+}};


			\draw [green] plot [smooth] coordinates {(7.8,9) (6.5,7) (6.1,0) (6.7,-5) (7.6,-7) (7.8,-8) (8,-8.8)};

		% 6V		
			\node (05) at (11,8.55) {};
			\node[red]  (15) at (10.6,8.50) {\textsf{+}};
			\node[red]  (25) at (9.60,8.00) {\textsf{+}};
			\node[red]  (35) at (9.40,7.50) {\textsf{+}};
			\node[red]  (45) at (8.80,7.00) {\textsf{+}};
			\node[red]  (55) at (8.55,6.50) {\textsf{+}};			
			\node[red]  (65) at (8.10,6.00) {\textsf{+}};
			\node[red]  (75) at (8.00,5.50) {\textsf{+}};
			\node[red]  (85) at (7.80,5.00) {\textsf{+}};
			\node[red]  (95) at (7.70,4.50) {\textsf{+}};
			\node[red] (105) at (7.70,4.00) {\textsf{+}};
			\node[red] (115) at (7.60,3.50) {\textsf{+}};
			\node[red] (125) at (7.50,3.00) {\textsf{+}};
			\node[red] (135) at (7.40,2.50) {\textsf{+}};
			\node[red] (145) at (7.40,2.00) {\textsf{+}};
			\node[red] (155) at (7.40,1.50) {\textsf{+}};
			\node[red] (165) at (7.40,1.00) {\textsf{+}};
			\node[red] (175) at (7.30,0.50) {\textsf{+}};
			\node[red] (185) at (7.40,0.00) {\textsf{+}};
			
			\node[red] (195) at (11.0,-7.50) {\textsf{+}};
			\node[red] (205) at (8.60,-7.00) {\textsf{+}};
			\node[red] (215) at (8.40,-6.50) {\textsf{+}};
			\node[red] (225) at (8.60,-6.00) {\textsf{+}};
			\node[red] (235) at (8.50,-5.50) {\textsf{+}};
			\node[red] (245) at (8.00,-5.00) {\textsf{+}};
			\node[red] (255) at (7.90,-4.50) {\textsf{+}};
			\node[red] (265) at (7.70,-4.00) {\textsf{+}};
			\node[red] (275) at (7.60,-3.50) {\textsf{+}};
			\node[red] (285) at (7.60,-3.00) {\textsf{+}};
			\node[red] (295) at (7.60,-2.50) {\textsf{+}};
			\node[red] (305) at (7.50,-2.00) {\textsf{+}};
			\node[red] (315) at (7.50,-1.50) {\textsf{+}};
			\node[red] (325) at (7.50,-1.00) {\textsf{+}};
			\node[red] (335) at (7.40,-0.50) {\textsf{+}};

			\draw [red] plot [smooth] coordinates {(11,8.55) (8.8,7) (7.6,4) (7.5,0) (7.6,-4) (9,-6.5) (11,-7.5)};
			
		% 7V		
			\node[purple] (25) at (8.60,0.00) {\textsf{+}};
			\node[purple] (26) at (10.0,-5.5) {\textsf{+}};

			\draw [purple] plot [smooth] coordinates {(11,6.3) (9.3,5) (8.7,3) (8.5,0) (8.6,-3) (9,-4.5) (10,-5.5) (11,-6)};

		% 8V		
			\node[orange] (25) at (10.0,0.00) {\textsf{+}};
			\node[orange] (26) at (10.5,-3.5) {\textsf{+}};

			\draw [orange] plot [smooth] coordinates {(11,4) (10.2,3) (10,0) (10.15,-3) (10.5,-3.5) (11,-3.8)};

		% Feldlinien
			\draw[- ,thick ,taupe] (0,-0.75) -- (11,-0.75);
			\draw[- ,thick ,taupe] (0,1.05) -- (11,1.05);
			\draw[- ,thick ,taupe] (0,1.35) -- (11,1.35);
			\draw[- ,thick ,taupe] (0,-2.85) -- (11,-2.85);
			\draw[- ,thick ,taupe] (0,-2.05) -- (11,-2.05);
			\draw[- ,thick ,taupe] (0,2.05) -- (11,2.05);

	\end{tikzpicture}
	\caption{Messwerte}
	\label{fig:Diagramm2}

	\end{figure}

	

\newpage
\section{Auswertung}
	\subsection{Betrag der elektrischen Feldst�rke $E$}
	Die ben�tigte Formel zu Berechnung der elektrischen Feldst�rke $E$ lautet:
	\[E=\frac{\text{Potentialdifferenz }\Delta\varphi}{\text{Abstand der Potentiallinien }\Delta s}\] 
	
	\subsubsection{Messreihe 1}
	Punkt $P_1 \left(\num{1.8} \mid 2\right)$,  Punkt $P_2 \left(\num{5,5} \mid \num{4,4}\right)$, Punkt $P_3 \left(\num{9.6} \mid 8\right)$:
		\[E_{P_1} = \frac{\SI{10}{\volt}}{\SI{0,11}{\metre}} = \SI{90,9091}{\volt \per \meter}\] 
		\[E_{P_2} = \frac{\SI{10}{\volt}}{\SI{0.115}{\metre}} = \SI{86,9565}{\volt \per \meter}\]
		\[E_{P_3} = \frac{\SI{10}{\volt}}{\SI{0.12}{\metre}} = \SI{83,3333}{\volt \per \meter}\]
	

	\subsubsection{Messreihe 2}
	Punkt $P_1 \left(\num{1,3} \mid \num{5,5}\right)$,  Punkt $P_2 \left(\num{6.2} \mid \num{1,35}\right)$, Punkt $P_3 \left(\num{7.38} \mid \num{-2,05}\right)$:
		\[E_{P_1} = \frac{\SI{10}{\volt}}{\SI{0,12}{\metre}} = \SI{83,3333}{\volt \per \meter}\] 
		\[E_{P_2} = \frac{\SI{10}{\volt}}{\SI{0.11}{\metre}} = \SI{90,9091}{\volt \per \meter}\]
		\[E_{P_3} = \frac{\SI{10}{\volt}}{\SI{0.11}{\metre}} = \SI{90,9091}{\volt \per \meter}\]

	\subsection{Beobachtung zum Metallring}
	In dem, in Messreihe 2 hinzugelegten, Metallring befand sich eine �quipotentialfl�che von \SI{4,3}{\volt}.

\end{document}








